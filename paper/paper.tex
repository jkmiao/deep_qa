\documentclass[11pt]{article}
\usepackage{acl2016}
\usepackage{amsmath}
\usepackage{times}
\usepackage{latexsym}
\usepackage{color}

\newcommand{\secref}[1]{Section~\ref{sec:#1}}
\newcommand{\figref}[1]{Figure~\ref{fig:#1}}
\newcommand{\algref}[1]{Algorithm~\ref{alg:#1}}
\newcommand{\tabref}[1]{Table~\ref{tab:#1}}
\newcommand{\mattnote}[1]{\textcolor{red}{NOTE: #1}}
\newcommand{\blank}{\_\_\_}
\newcommand{\predicate}[1]{\ensuremath{\textsc{#1}}}
\newcommand{\entity}[1]{\ensuremath{\mathrm{#1}}}

\newcommand{\true}[0]{\predicate{true}}
\newcommand{\false}[0]{\predicate{false}}

%\aclfinalcopy % Uncomment this line for the final submission
%\def\aclpaperid{***} %  Enter the acl Paper ID here

\title{Combining formal and distributional models of predicate semantics}

\author{}%Matt Gardner and Jayant Krishnamurthy\\
%Allen Institute for Artificial Intelligence\\
%Seattle, Washington, USA\\
%{\tt \{mattg,jayantk\}@allenai.org}}

\date{}

\begin{document}

\maketitle

\begin{abstract}

  We consider the problem of learning models of predicate (category and
  relation) semantics using both distributional information and structured
  information contained in a formal knowledge base.  Most previous approaches
  to modeling predicate semantics have either been purely distributional (such
  as word2vec and knowledge base embedding methods) or purely formal (such as
  most semantic parsers).  Distributional approaches have very broad coverage,
  but lack the compositional, logical semantics used by semantic parsers, and
  it has so far been difficult to combine the benefits of both of these
  techniques.

  In this paper, we present a method that incorporates logical statements
  derived from a knowledge base into distributional models of categories and
  relations.  Instead of mapping a phrase to a single logical statement, as
  done by semantic parsers, our method uses logical statements as features
  added to distributional models of words, allowing the model to learn that
  Freebase entities that are "honchos" are people, not buildings, and that they
  are frequently CEOs or film producers.  We use these models in an
  open-vocabulary semantic parsing task, showing a 63\% relative improvement in
  mean average precision over a distributional-only model, and a X\%
  improvement over a formal-only approach.

\end{abstract}

\section{Introduction}

How should a computer program represent the meaning of a word or a phrase?  The
answer to this question has a long and rich history.  Motivated by
Harris's distributional
hypothesis~\cite{harris-1954-distributional-hypothesis}, much work has been put
into representing words as collections of the contexts in which they appear.
Recent work with neural networks and word embeddings have shown that these
representations can be very useful for a number of natural language processing
tasks. \mattnote{cite something here}

At the same time, other researchers have found Montague's ideas about English
as a formal language to be useful~\cite{montague-1974-english-formal-language}.
Modern semantic parsers make use of a model-theoretic semantics that represents
words and phrases as formal predicates.  These predicates are then typically
mapped onto a query over a structured knowledge base, such as
Freebase~\cite{freebase-2008-bollacker}.  This kind of modeling allows semantic
parsers to handle the compositional structure of natural language and to
leverage large, curated knowledge sources to perform various natural language
processing tasks.  \mattnote{cite some semantic parsing papers}

Both of these approaches to modeling predicate semantics have aspects that are
very appealing.  It is relatively straightforward to learn a distributional
vector for every word or phrase in a corpus, allowing easy generalization from
words labeled with part of speech tags or dependency information to other words
that are distributionally similar.  Yet distributional approaches struggle to
make good use of the structured information contained in curated knowledge
sources like Freebase and WordNet~\cite{wordnet-1995}.  On the other hand,
because formal approaches to semantics typically model word meaning in terms of
a structured knowledge source, their applicability is limited to the set of
concepts represented in that knowledge source---a semantic parser trained with
Freebase cannot even represent words whose meaning has no correlate in the
Freebase ontology, such as ``front-runner'' and ``honcho''.

In this work we introduce models of predicate semantics that combine the
benefits of both distributional and formal approaches.  While there is no
Freebase predicate that corresponds exactly to the meaning of ``honcho'',
Freebase \emph{does} contain many predicates that can \emph{partially}
represent the word.  Our models extract features from Freebase that are
correlated with words and phrases seen in natural language text, and learns
weights for those features as additional components of a distributional model
of the word's meaning.

To demonstrate these models, we focus on the task of fill-in-the-blank natural
language queries, such as ``Italian architect \blank{}''.  Similar to many
current semantic parsers, we first convert this natural language query into a
logical form: $\lambda(\entity{x}) \predicate{architect}(\entity{x}) \land
\predicate{architect\_N/N}(\entity{Italian}, \entity{x})$.  A typical semantic
parser (e.g., that of Kwiatkowski et
al.~\shortcite{kwiatkowski-2013-ontology-matching}) would then try to map this
logical form to a Freebase query, matching \entity{Italian} to
(\predicate{/people/person/nationality}, \entity{Italy}) and
\predicate{architect} to (\predicate{/type/object/type},
\entity{/architecture/architect}).  In contrast, we follow Krishnamurthy and
Mitchell~\shortcite{krishnamurthy-2015-semparse-open-vocabulary} in learning a
model for each predicate that can give a \emph{score} to potential arguments.
However, while the models of Krishnamurthy and Mitchell are purely
distributional, using Freebase only as a means of finding entities in text, we
inject additional information from Freebase into the models for each predicate.
The model we learn for \predicate{architect} has a \emph{feature} capturing the
fact that an entity has the type \entity{/architecture/architect} in Freebase,
and the model for \predicate{architect\_N/N} has a \emph{feature} capturing
whether the first argument is the country of nationality of the second
argument.

This is a fundamentally new kind of predicate semantics.  It is
distributional---part of the model is still trained in a manner very similar to
word2vec---yet it is also formal.  However, instead of committing to a single
Freebase query to represent a phrase, as traditional formal approaches do, our
models are flexible, encoding many possible Freebase queries as features that
can give hints to the model.  Thus words that would be impossible to represent
as a single Freebase query can still benefit from the information contained in
Freebase---our model for ``front-runner'' learns that acceptable entities are
politicians, even though there is no exact encoding of ``front-runner'' in
Freebase.

The rest of this paper is organized as follows.  In \secref{background}, we
describe the relevant background necessary to understand the models we propose,
and we situate our model in the context of related work.  In \secref{method},
we give a formal description of our models of predicate semantics.  In
\secref{evaluation}, we present experimental results on fill-in-the-blank
natural language query task.  Using web text linked to Freebase as our training
and test data, we show a 63\% relative improvement over prior work, giving a
new state-of-the-art result on this dataset.  In \secref{discussion}, we give a
detailed discussion of the benefits and drawbacks of the models we have
proposed.  We then conclude in \secref{conclusion}.

\section{Background}
\label{sec:background}

Our proposed method is best viewed as the addition of SFE
features~\cite{gardner-2015-sfe} to Krishnamurthy and
Mitchell's~\shortcite{krishnamurthy-2015-semparse-open-vocabulary}
open-vocabulary semantic parser.  Accordingly, in this section we briefly
describe both of these techniques, as well as give a short overview of other
closely related work.

\subsection{Open-vocabulary semantic parsing}
\label{sec:jayant-semparse}


This system predicts a denotation $\gamma$, i.e., a set of Freebase
entities, for a natural language text $s$, using the following
probabilistic model:

\begin{align*}
P(\gamma | s) & = \sum_w \sum_\ell P(\gamma | \ell, w) P(w ; \theta, \phi) P(\ell | s)
\end{align*}

In this equation, $P(\ell | s)$ represents a semantic parser that
produces a logical form $\ell$ given the text $s$. This logical form
contains open-vocabulary predicates, as in the ``architect'' example
above. This factor is deterministic, meaning that it assigns
probability one to a single logical form. The second factor $P(w)$
represents a distribution over possible worlds $w$, where a world is
an assignment of truth values to all possible predicate
instances. This distribution can be efficiently represented using a
probabilistic database because each predicate instance is treated as
an independent random variable. The truth probabilities of these
variables are predicted using a matrix factorization approach similiar
to that of Riedel et al., \shortcite{riedel-2013-mf-universal-schema};
$\theta$ and $\phi$ represent the parameters of the matrix
factorization. Finally, $P(\gamma | \ell, w)$ represents the
deterministic evaluation of the logical form $\ell$ on the world $w$
to produce a denotation $\gamma$. This factor is similar to other
semantic parsing work, where it represents the deterministic
evaluation of a query against a given database. Algorithms from the
probabilistic databases community enable us to efficiently compute the
marginal probability that an entity $e$ is in the denotation of a
sentence, $\sum_\gamma P(e \in \gamma | s)$, for a large number of
queries; we refer the reader to (TODO).

The training problem in this model is to learn the parameters $\theta$
and $\phi$ of the probabilistic database. The probability of a
category instance $c(e)$ and relation instance $r(e_1, e_2)$ in the
database is given by:

\begin{align*}
  P(c(e) & = \true{}) & = & \sigma ( \theta_c^T \phi_e ) \\
  P(r(e_1, e_2) & = \true{}) & = & \sigma ( \theta_r^T \phi_{(e_1, e_2)} )
\end{align*}

The parameters $\theta$ represent $k$-dimensional embeddings of each
category and relation predicate, and the parameters $\phi$ represent
$k$-dimensional embeddings of each Freebase entity and entity
pair. The probability of a predicate instance is the inner product of
the corresponding predicate and entity embeddings. Thus, predicates
with nearby embeddings will have similar distributions over the
entities in their denotation.

The parameters $\theta$ and $\phi$ are learned from predicate/entity
co-occurrence counts collected from a large corpus. The training data
consists of query/entity pairs, $\{(\ell^i, e^i)\}$, where the entity
$e^i$ is observed to be an element of $\ell^i$'s denotation. These
pairs are collected by semantically parsing entity-linked sentences
and simplifying the resulting logical forms to identify conjunctions
of predicates where all arguments are bound to specific
entities. Extracting an entity from this conjunction results in a
query/entity pair. The model is trained using a ranking objective that
learns to rank the observed entities for each query above unobserved
entities.

\subsection{Subgraph feature extraction}

Subgraph feature extraction (SFE) is a technique introduced by Gardner and
Mitchell~\shortcite{gardner-2015-sfe} for generating feature matrices over node
pairs in a graph with labeled edges.  Given a pair of nodes in a graph, SFE
performs a search to characterize a subgraph around those two nodes, then
extracts various kinds of features from that subgraph, such as the set of edge
sequences that connect the two nodes.  \mattnote{Give an example graph with
features.}  Gardner and Mitchell used these features to perform link
prediction in the graph, using characteristics of the subgraph around two nodes
to determine if an edge of a particular type should exist between those nodes
(when the graph corresponds to a knowledge base like Freebase, this task is
also known as knowledge base completion). Gardner and Mitchell only used these
features to model formal KB relations, however.

We differ from the work done by Gardner and Mitchell in two ways.  First, we
apply these features to models of natural language, instead of formal KB
relations.  There are some non-trivial problems that need to be solved in order
to make this work, which will be discussed in \secref{method}.  Second, Gardner
and Mitchell only considered generating feature matrices over node \emph{pairs}
in a graph, with which one can model binary relations.  The natural language
queries we wish to answer also involve unary predicates, or categories.  We
thus extend SFE to also generate feature matrices over \emph{nodes} in a graph,
an extension that is straightforward and is also discussed in more detail in
\secref{method}.

\subsection{Other related work}

Some related work on including structured information in distributional
representations: Manaal's retrofitting semantic lexicons, Sebastian Riedel's
``Injecting Logical Background Knowledge into Embeddings for Relation
Extraction'', work on PRA and KBC more generally\ldots Also, the whole line of
work on knowledge base embeddings (RESCAL, TransE, etc.) is at least slightly
relevant here, as is PRA.  \mattnote{Actually write
this section}.

\section{Method}
\label{sec:method}

\subsection{Adding SFE features to Jayant's models}
\label{sec:formal-and-distributional}

\mattnote{TODO}

\subsection{Finding relevant features for each predicate}
\label{sec:computing-pmi}

\mattnote{TODO}

\subsection{Improved logical forms}
\label{sec:better-lfs}

\mattnote{TODO}

\section{Evaluation}
\label{sec:evaluation}

We evaluate our models of predicate semantics on a fill-in-the-blank natural
language query task.  Each test example is a natural language phrase containing
at least two Freebase entities, one of which is held out.  The system must
propose Freebase entities to fill in the blank left by the held out entity, and
the predicted entities are then judged manually for correctness.  We compare
our proposed models, which combine distributional and formal elements, with
distributional-only and formal-only baselines.

\subsection{Data}

We use the dataset introduced by Krishnamurthy and
Mitchell~\shortcite{krishnamurthy-2015-semparse-open-vocabulary}, which
consists of the ClueWeb09 web
corpus\footnote{http://www.lemuproject.org/clueweb09.php} along with Google's
FACC entity linking of that corpus to
Freebase~\cite{gabrilovich-2013-clueweb-entity-linking}.  3 million webpages
from this corpus were processed with a CCG parser to produce logical forms, as
described in \secref{jayant-semparse}.  This produced 2.1m predicate instances
involving 142k entity pairs and 184k entities.  After removing
infrequently-seen predicates (seen fewer than 6 times), there were 25k
categories and 4.2k relations.\footnote{The differences in numbers reported
here versus those reported by Krishnamurthy and
Mitchell~\shortcite{krishnamurthy-2015-semparse-open-vocabulary} are due to the
improved logical form generation, discussed in \secref{better-lfs}.}

\subsection{Methodology}

\subsection{Models}

\subsection{Results}

\section{Discussion}
\label{sec:discussion}

\mattnote{TODO}

\section{Conclusion}
\label{sec:conclusion}

\mattnote{TODO}

%\section*{Acknowledgments}

\bibliography{bib}
\bibliographystyle{acl2016}

\end{document}
